\documentclass[12pt,letterpaper]{article}
\usepackage{graphicx,textcomp}
\usepackage{natbib}
\usepackage{setspace}
\usepackage{fullpage}
\usepackage{color}
\usepackage[reqno]{amsmath}
\usepackage{amsthm}
\usepackage{fancyvrb}
\usepackage{amssymb,enumerate}
\usepackage[all]{xy}
\usepackage{endnotes}
\usepackage{lscape}
\newtheorem{com}{Comment}
\usepackage{float}
\usepackage{hyperref}
\newtheorem{lem} {Lemma}
\newtheorem{prop}{Proposition}
\newtheorem{thm}{Theorem}
\newtheorem{defn}{Definition}
\newtheorem{cor}{Corollary}
\newtheorem{obs}{Observation}
\usepackage[compact]{titlesec}
\usepackage{dcolumn}
\usepackage{tikz}
\usetikzlibrary{arrows}
\usepackage{multirow}
\usepackage{xcolor}
\newcolumntype{.}{D{.}{.}{-1}}
\newcolumntype{d}[1]{D{.}{.}{#1}}
\definecolor{light-gray}{gray}{0.65}
\usepackage{url}
\usepackage{listings}
\usepackage{color}
\usepackage{subcaption} 
\usepackage{mdframed} % for creating framed boxes
\usepackage{fancyhdr}
\pagestyle{fancy}
\usepackage{adjustbox} % for resizing tables
\usepackage{rotating}
\usepackage{booktabs}

\fancyhf{}
\fancyfoot[R]{\thepage} % Place page number at the bottom-right corner
\renewcommand{\headrulewidth}{0pt} % Remove header rule
\renewcommand{\footrulewidth}{0pt} % Remove footer rule

\definecolor{codegreen}{rgb}{0,0.6,0}
\definecolor{codegray}{rgb}{0.5,0.5,0.5}
\definecolor{codepurple}{rgb}{0.58,0,0.82}
\definecolor{backcolour}{rgb}{0.95,0.95,0.92}

\lstdefinestyle{mystyle}{
	backgroundcolor=\color{backcolour},   
	commentstyle=\color{codegreen},
	keywordstyle=\color{magenta},
	numberstyle=\tiny\color{codegray},
	stringstyle=\color{codepurple},
	basicstyle=\footnotesize,
	breakatwhitespace=false,         
	breaklines=true,                 
	captionpos=b,                    
	keepspaces=true,                 
	numbers=left,                    
	numbersep=5pt,                  
	showspaces=false,                
	showstringspaces=false,
	showtabs=false,                  
	tabsize=2
}
\lstset{style=mystyle}
\newcommand{\Sref}[1]{Section~\ref{#1}}
\newtheorem{hyp}{Hypothesis}

\title{Replication}
\date{\today}
\author{Kaley Burg}


\begin{document}
	\maketitle
	\subsection*{1. Paper Title and Information}
	\begin{itemize}
		\item \textbf{Title:} Exclusion and Cooperation in Diverse Societies: Experimental Evidence from Israel
		\item \textbf{Authors:} RYAN D. ENOS Harvard University NOAM GIDRON Hebrew University of Jerusalem and Princeton University
		\item \textbf{Journal and Year:} American Political Science Review (2018)
		\item \textbf{Research Question and Design:}
			\begin{itemize}
				\item Do exclusionary preferences stop individuals from cooperating with one another?
				\item Does the lack of cooperation therefore prevent societies from providing public goods?
				\item \textbf{Main question:} How do exclusionary attitudes among the Jewish majority predict discriminatory behaviour towards Palestinian citizens in Israel?
				\item Authors also aim to understand how exclusionary attitudes are distinct from from behaviors.
				\item \textbf{Data:}
				\item Uses lab-in-the-field data to study cooperation and exclusionary attitudes of Jewish people towards Palestinian Citizens of Israel (PCI)
				\item Measures cooperation (behaviours) with an economic decision making game
				\item Measures exclusionary attitudes with a social distance scale which asks participants to choose the degree of proximity that they would accept an outgroup member. Responses range from family relative to no relationship.
				\item \textbf{Variables:}
				\item arab\_accept, arab\_accept\_character, arab\_reject\_binary - all variables corresponding to the level of exclusionary attitudes towards PCI, either in categorical form, continuous numeric, or binary
				\item age - continuous age variable
				\item foreign\_born - binary variable, 1 corresponding to foreign born 
				\item sex - binary variable, 1 corresponding to male
				\item left\_right - ordinal variable, 1 meaning far left to 7 meaning far right             
				\item religiosity - categorical data with levels anti, orthodox, secular, traditional, ultra-orthodox (treated as ordinal in the manuscript but treated as categorical dummies in all analysis)
				\item education - ordinal variable with levels graduate, high school, primary, and undergraduate
				\item income - ordinal with levels average, high, low, very high, very low 
				\item ethnicity 
				\item arab\_interactions - level of interaction with arabs with levels daily, monthly, never, week, and yearly
			\end{itemize}
		\item \textbf{Hypothesis:}
			\begin{itemize}
				\item \textbf{Note:} The authors do not explicitly state their hypotheses. Instead they state their contributions to the literature prior to their results section. I will be using this to infer their hypotheses
					\begin{enumerate}
						\item Levels of exclusionary attitudes are high among the Jewish majority towards PCI
						\item These levels of exclusion are highest for lower-status Jews (low SES, uneducated, and the ultra-Orthodox population)
						\item Exclusionary attitudes are symbolic, meaning they are stable and affect other attitudes.
						\item Cooperation is informed/predicted by exclusionary attitiudes preferences for exclusion. In other words, those with greater exclusionary attitudes towards PCI are also less likely to cooperate with PCI.
						\item The attitudes-behaviour connection is not moderated by other factors such as perceptions of Arabs’ trustworthiness.
					\end{enumerate}
			\end{itemize}
		
	\end{itemize}

	
	\vspace{.25cm}
	\subsection*{2. Replicate the figures and tables of the main findings found in the manuscript}
	\begin{itemize}
		\item Below is all the code used to replicate these figures and tables
		\lstinputlisting[language=R, firstline=16, lastline=217]{KB_rep_cleaned.R}
	\end{itemize}
	\newpage
	\subsubsection{Figures}
\begin{figure}[htbp]
	\begin{mdframed}[
		skipabove=0.5\baselineskip, % adjust the space above the framed box
		skipbelow=0.5\baselineskip, % adjust the space below the framed box
		linewidth=1pt, % set the width of the frame lines
		innerleftmargin=10pt, % adjust the left margin of the frame
		innerrightmargin=10pt, % adjust the right margin of the frame
		innertopmargin=10pt, % adjust the top margin of the frame
		innerbottommargin=10pt, % adjust the bottom margin of the frame
		]
		\centering
		\caption{Social Distance by Group}
		\label{fig:Social Distance by Group}
		\begin{subfigure}[b]{0.4\textwidth} % adjust the width as needed
			\includegraphics[width=\textwidth]{distance_uo_arabs}
			\caption{Distance UO Arabs}
			\label{fig:distance_uo_arabs}
		\end{subfigure}
		\hfill
		\begin{subfigure}[b]{0.4\textwidth} % adjust the width as needed
			\includegraphics[width=\textwidth]{distance_secular_arabs}
			\caption{Distance Secular Arabs}
			\label{fig:distance_secular_arabs}
		\end{subfigure}
	\end{mdframed}
\end{figure}
	
	
\begin{figure}[htbp]
	\begin{mdframed}[
		skipabove=0.5\baselineskip, % adjust the space above the framed box
		skipbelow=0.5\baselineskip, % adjust the space below the framed box
		linewidth=1pt, % set the width of the frame lines
		innerleftmargin=10pt, % adjust the left margin of the frame
		innerrightmargin=10pt, % adjust the right margin of the frame
		innertopmargin=10pt, % adjust the top margin of the frame
		innerbottommargin=10pt, % adjust the bottom margin of the frame
		]
		\centering
		\caption{Religiosity, ideology, education, and social distance}
		\label{fig:Religiosity_ideology_education_social_distance}
		\begin{subfigure}[b]{0.3\textwidth} % adjust the width as needed
			\centering
			\includegraphics[width=0.8\textwidth]{predict_religiosity} % adjust the scale of the image
			\caption{Religiosity}
			\label{fig:predict_religiosity}
		\end{subfigure}\hfill
		\begin{subfigure}[b]{0.3\textwidth} % adjust the width as needed
			\centering
			\includegraphics[width=0.8\textwidth]{predict_left_right} % adjust the scale of the image
			\caption{Political Ideology}
			\label{fig:predict_leftright}
		\end{subfigure}\hfill
		\begin{subfigure}[b]{0.3\textwidth} % adjust the width as needed
			\centering
			\includegraphics[width=0.8\textwidth]{predict_education} % adjust the scale of the image
			\caption{Education}
			\label{fig:predict_education}
		\end{subfigure}
	\end{mdframed}
\end{figure}
		
	\subsubsection{Tables}
	
% Table created by stargazer v.5.2.3 by Marek Hlavac, Social Policy Institute. E-mail: marek.hlavac at gmail.com
% Date and time: Sat, Mar 30, 2024 - 18:03:06
\begin{table}[!htbp] \centering 
	\footnotesize
	\setlength{\tabcolsep}{4pt} % reducing padding
	\caption{} 
	\label{} 
	\begin{tabular}{@{\extracolsep{5pt}}lcc} 
		\\[-1.8ex]\hline 
		\hline \\[-1.8ex] 
		& \multicolumn{2}{c}{\textit{Dependent variable:}} \\ 
		\cline{2-3} 
		\\[-1.8ex] & \multicolumn{2}{c}{Social Distance with PCI} \\ 
		\\[-1.8ex] & (1) & (2)\\ 
		\hline \\[-1.8ex] 
		Age & $-$0.003 & $-$0.004 \\ 
		& (0.005) & (0.005) \\ 
		Foreign Born & $-$0.326$^{*}$ & $-$0.246 \\ 
		& (0.197) & (0.188) \\ 
		Sex & $-$0.519$^{***}$ & $-$0.376$^{***}$ \\ 
		& (0.149) & (0.144) \\ 
		Left-right & 0.314$^{***}$ & 0.276$^{***}$ \\ 
		& (0.055) & (0.052) \\ 
		Secular & $-$0.425$^{*}$ & $-$0.353 \\ 
		& (0.256) & (0.246) \\ 
		Traditional & 0.236 & 0.287 \\ 
		& (0.285) & (0.272) \\ 
		Ultra Orthodox & 0.650$^{***}$ & 0.495$^{**}$ \\ 
		& (0.241) & (0.232) \\ 
		High-school & 0.368 & 0.250 \\ 
		& (0.337) & (0.321) \\ 
		Primary-school & 1.013$^{**}$ & 0.634 \\ 
		& (0.466) & (0.458) \\ 
		Undergrad & 0.570$^{*}$ & 0.430 \\ 
		& (0.340) & (0.325) \\ 
		High income & $-$0.474$^{*}$ & $-$0.384 \\ 
		& (0.270) & (0.260) \\ 
		Low income & $-$0.096 & $-$0.180 \\ 
		& (0.216) & (0.208) \\ 
		Very high income & 0.151 & $-$0.108 \\ 
		& (0.402) & (0.387) \\ 
		Very low income & $-$0.002 & 0.048 \\ 
		& (0.202) & (0.195) \\ 
		Mixed & $-$0.358 & $-$0.478$^{*}$ \\ 
		& (0.283) & (0.269) \\ 
		Other & 0.232 & $-$0.186 \\ 
		& (0.456) & (0.438) \\ 
		Sephardic & 0.132 & 0.137 \\ 
		& (0.163) & (0.157) \\ 
		Month &  & 0.931$^{***}$ \\ 
		&  & (0.296) \\ 
		Never &  & 1.486$^{***}$ \\ 
		&  & (0.259) \\ 
		Week &  & 0.469 \\ 
		&  & (0.311) \\ 
		Year &  & 1.166$^{***}$ \\ 
		&  & (0.312) \\ 
		Constant & 3.846$^{***}$ & 3.046$^{***}$ \\ 
		& (0.556) & (0.560) \\ 
		\hline \\[-1.8ex] 
		Observations & 375 & 372 \\ 
		R$^{2}$ & 0.289 & 0.370 \\ 
		Adjusted R$^{2}$ & 0.255 & 0.332 \\ 
		Residual Std. Error & 1.345 (df = 357) & 1.277 (df = 350) \\ 
		F Statistic & 8.537$^{***}$ (df = 17; 357) & 9.785$^{***}$ (df = 21; 350) \\ 
		\hline 
		\hline \\[-1.8ex] 
		\textit{Note:}  & \multicolumn{2}{r}{$^{*}$p$<$0.1; $^{**}$p$<$0.05; $^{***}$p$<$0.01} \\ 
	\end{tabular} 
\end{table} 




% Table created by stargazer v.5.2.3 by Marek Hlavac, Social Policy Institute. E-mail: marek.hlavac at gmail.com
% Date and time: Sat, Mar 30, 2024 - 18:22:19
\begin{table}[!htbp] \centering 
	\small
	\caption{} 
	\label{} 
	\begin{tabular}{@{\extracolsep{5pt}}lcccc} 
		\\[-1.8ex]\hline 
		\hline \\[-1.8ex] 
		& \multicolumn{4}{c}{\textit{Dependent variable:}} \\ 
		\cline{2-5} 
		\\[-1.8ex] & \multicolumn{4}{c}{Cooperation with PCI (=1)} \\ 
		\\[-1.8ex] & (1) & (2) & (3) & (4)\\ 
		\hline \\[-1.8ex] 
		Social Distance (binary) & $-$0.820$^{***}$ & $-$0.631$^{**}$ & $-$0.639$^{**}$ & $-$0.579$^{*}$ \\ 
		& (0.243) & (0.306) & (0.264) & (0.319) \\ 
		Age &  & $-$0.002 &  & $-$0.002 \\ 
		&  & (0.008) &  & (0.008) \\ 
		Foreign Born &  & 0.137 &  & 0.053 \\ 
		&  & (0.318) &  & (0.325) \\ 
		Sex &  & 0.318 &  & 0.286 \\ 
		&  & (0.245) &  & (0.251) \\ 
		Left-right &  & $-$0.208$^{**}$ &  & $-$0.198$^{**}$ \\ 
		&  & (0.093) &  & (0.095) \\ 
		Secular &  & $-$0.916$^{**}$ &  & $-$1.097$^{***}$ \\ 
		&  & (0.411) &  & (0.424) \\ 
		Traditional &  & $-$0.728 &  & $-$0.919$^{*}$ \\ 
		&  & (0.459) &  & (0.470) \\ 
		Ultra Orthodox &  & $-$0.357 &  & $-$0.501 \\ 
		&  & (0.386) &  & (0.394) \\ 
		High-school &  & 0.826 &  & 0.974 \\ 
		&  & (0.583) &  & (0.604) \\ 
		Primary school &  & 0.934 &  & 1.060 \\ 
		&  & (0.773) &  & (0.792) \\ 
		Undergrad &  & 0.469 &  & 0.604 \\ 
		&  & (0.590) &  & (0.613) \\ 
		High &  & $-$0.085 &  & $-$0.036 \\ 
		&  & (0.435) &  & (0.441) \\ 
		Low &  & $-$0.409 &  & $-$0.430 \\ 
		&  & (0.348) &  & (0.353) \\ 
		Very high &  & 0.158 &  & 0.086 \\ 
		&  & (0.625) &  & (0.632) \\ 
		Very low &  & $-$0.356 &  & $-$0.326 \\ 
		&  & (0.323) &  & (0.326) \\ 
		Mixed &  & $-$0.135 &  & $-$0.137 \\ 
		&  & (0.456) &  & (0.460) \\ 
		Other &  & 0.593 &  & 0.998 \\ 
		&  & (0.715) &  & (0.752) \\ 
		Sephardic &  & $-$0.312 &  & $-$0.262 \\ 
		&  & (0.267) &  & (0.270) \\ 
		Trust in Arabs &  &  & 0.536$^{*}$ & 0.419 \\ 
		&  &  & (0.291) & (0.341) \\ 
		Constant & $-$0.045 & 1.035 & $-$0.271 & 0.867 \\ 
		& (0.213) & (0.905) & (0.249) & (0.942) \\ 
		\hline \\[-1.8ex] 
		Observations & 439 & 375 & 432 & 371 \\ 
		Log Likelihood & $-$274.275 & $-$226.808 & $-$268.588 & $-$222.142 \\ 
		Akaike Inf. Crit. & 552.549 & 491.617 & 543.176 & 484.285 \\ 
		\hline 
		\hline \\[-1.8ex] 
		\textit{Note:}  & \multicolumn{4}{r}{$^{*}$p$<$0.1; $^{**}$p$<$0.05; $^{***}$p$<$0.01} \\ 
	\end{tabular} 
\end{table} 







\newpage
\textbf{}

	\subsection*{3. Insert some twist based on your gained knowledge from Stats I and II}
	
	\begin{itemize}
		\item For my twist, I first checked the model-fit of the ordered multinomial regression that the authors included in their replication code. Then, I ran an unordered model as well.
		\subsubsection{Ordered Model Checks}
			\begin{itemize}
				\item \textbf{NOTE:} The authors did not include this table in their main manuscript or in their appendix materials. They do include a footnote though that states "Ordered logit regression provides substantively similar results." (pg. 749).
			\end{itemize} 
		\item The code that the authors included is as follows: 
		\lstinputlisting[language=R, firstline=290, lastline=299]{KB_rep_cleaned.R}
			\begin{itemize}
				\item Note that it is their comment in the script that states that the results are substantively similar to lm
				\item There was no further justification of this from the authors
			\end{itemize}
		\item First, I made sure that this model was run properly by creating my own ordered factor variable and running it again
		\lstinputlisting[language=R, firstline=303, lastline=320]{KB_rep_cleaned.R}
		\item I found that the results are the same, shown in Table 3
		\end{itemize}
		
		
		
			% Table created by stargazer v.5.2.3 by Marek Hlavac, Social Policy Institute. E-mail: marek.hlavac at gmail.com
		% Date and time: Sat, Mar 30, 2024 - 19:00:02
		\begin{table}[p] \centering 
			\caption{}
			\footnotesize 
			\setlength{\tabcolsep}{1pt} % reducing padding
			\label{} 
			\renewcommand{\arraystretch}{0.9} % Default value: 1
			\begin{tabular}{@{\extracolsep{1pt}}lcc} 
				\\[-1.8ex]\hline 
				\hline \\[-1.8ex] 
				& \multicolumn{2}{c}{\textit{Dependent variable:}} \\ 
				\cline{2-3} 
				\\[-1.8ex] & arab\_acceptTEST & arab\_accept \\ 
				\\[-1.8ex] & (1) & (2)\\ 
				\hline \\[-1.8ex] 
				age & $-$0.007 & $-$0.007 \\ 
				& (0.007) & (0.007) \\ 
				& & \\ 
				foreign\_born1 & $-$0.426 & $-$0.426 \\ 
				& (0.275) & (0.275) \\ 
				& & \\ 
				as.factor(as.character(religiosity))secular & $-$0.617$^{*}$ & $-$0.617$^{*}$ \\ 
				& (0.360) & (0.360) \\ 
				& & \\ 
				as.factor(as.character(religiosity))trad & 0.225 & 0.225 \\ 
				& (0.409) & (0.409) \\ 
				& & \\ 
				as.factor(as.character(religiosity))u\_orthodox & 0.849$^{**}$ & 0.849$^{**}$ \\ 
				& (0.354) & (0.354) \\ 
				& & \\ 
				sex1 & $-$0.778$^{***}$ & $-$0.778$^{***}$ \\ 
				& (0.219) & (0.219) \\ 
				& & \\ 
				as.factor(education)high & 0.264 & 0.264 \\ 
				& (0.485) & (0.485) \\ 
				& & \\ 
				as.factor(education)primary & 1.211$^{*}$ & 1.211$^{*}$ \\ 
				& (0.702) & (0.702) \\ 
				& & \\ 
				as.factor(education)undergrad & 0.445 & 0.445 \\ 
				& (0.490) & (0.490) \\ 
				& & \\ 
				as.factor(income)high & $-$0.606 & $-$0.606 \\ 
				& (0.372) & (0.372) \\ 
				& & \\ 
				as.factor(income)low & $-$0.018 & $-$0.018 \\ 
				& (0.307) & (0.307) \\ 
				& & \\ 
				as.factor(income)v\_high & 0.010 & 0.010 \\ 
				& (0.550) & (0.550) \\ 
				& & \\ 
				as.factor(income)v\_low & 0.059 & 0.059 \\ 
				& (0.290) & (0.290) \\ 
				& & \\ 
				left\_right & 0.406$^{***}$ & 0.406$^{***}$ \\ 
				& (0.082) & (0.082) \\ 
				& & \\ 
				as.factor(ethnicity)mixed & $-$0.403 & $-$0.403 \\ 
				& (0.404) & (0.404) \\ 
				& & \\ 
				as.factor(ethnicity)other & 0.250 & 0.250 \\ 
				& (0.612) & (0.612) \\ 
				& & \\ 
				as.factor(ethnicity)sep & 0.213 & 0.213 \\ 
				& (0.240) & (0.240) \\ 
				& & \\ 
				\hline \\[-1.8ex] 
				Observations & 375 & 375 \\ 
				\hline 
				\hline \\[-1.8ex] 
				\textit{Note:}  & \multicolumn{2}{r}{$^{*}$p$<$0.1; $^{**}$p$<$0.05; $^{***}$p$<$0.01} \\ 
			\end{tabular} 
		\end{table} 
		
	
		
		
		\newpage
		
	\begin{itemize}
		\item Then I checked with categories corresponded to each number in the arab\_accept factor and made sure that this matched the arab\_accept\_character factor
		\lstinputlisting[language=R, firstline=340, lastline=340]{KB_rep_cleaned.R}
		\item I found that relative is coded as 1, friend as 2, neighbor as 3, coworker as 4, citizen as 5, visitor as 6, and none as 7
		\item So, the further an individual is along the latent variable, the more prejudiced they are towards PCR.
		\item Now that I confirmed that their model was run correctly, I ran several other tests to check whether this model would have been more appropriate than both a linear regression model and an unordered multinomial regression.
		\item First, I checked the intercept values and plotted them
		\lstinputlisting[language=R, firstline=380, lastline=380]{KB_rep_cleaned.R}
\begin{figure}[htbp]
	\centering
	\includegraphics[width=1\textwidth]{intercepts.png} % Adjust the width as needed
	\caption{Intercepts} % Add a caption if needed
	\label{fig:intercepts}
\end{figure}

		\item The reason I did this was to check the distance between each of the intercept values
		\item Long (1997) states that "in general, the results of the LRM only correspond to those of the ORM \textit{if} the thresholds are all about the same distance apart. When this is not the case, the LRM can give very misleading results. 
		\item Aside from theoretical reasons about the outcome variable not being continuous and therefore not being a good fit for a linear regression model, this also serves as a first indication that the linear regression model might be giving skewed results.
		\item Next, I checked the author's assumptions about whether the ordered logit model  produces similar results to the linear model.
		\item Referring back to Table 3, using either of the models since they produce the same results, we can see that the following variables are significant at the $p<0.001$ level: sex/gender, and political ideology (left/right). The following are significant at the $p<0.05$ level: ultra-Orthodox religious identity (compared to orthodox as the reference category). The following are significant at the $p<0.01$ level: secular religious identity (compared to orthodox as the reference category), and primary education level (compared to grad).
		\item Comparing this to the results from the linear model (Table 1, left column), we see that the following variables are significant at the $p<0.001$ level: sex/gender, political ideology (left/right), and ultra-Orthodox (compared to Orthodox). The following are significant at the $p<0.05$ level: primary educational level (compared to grad as reference category). The following are significant at the $p<0.01$ level: foreign born (compared to not foreign born as the reference category), secular religious identity (compared to orthodox as reference category), undergraduate educational level (compared to grad as reference category), and high income (compared to average as the reference category) and primary education level (compared to grad).
		\item We can see from these results that, although the sex, political ideology, ultra-orthodox, secular, and primary education variables are signficant in both models (although at varying levels), the linear model contains significance for the variables foreign born, undergraduate, and high income as well.
		\item Although the authors did not state how they were defining substantively similar, I am going to assume that they consider this result to be similar enough to use OLS as the primary model.  
		\item I also checked the p-values by converting the results to odd-ratio form and constructing confidence intervals at the 95\% level for both the log odds and odd ratios in the ordered multinomial model
		\lstinputlisting[language=R, firstline=396, lastline=447]{KB_rep_cleaned.R}	
		\item The results, shown in Tables 4 and 5, indicate that at the $p<0.05$ level are: ultra-Orthodox, sex, and political ideology.
		\item This is used as an added check to make sure our results are in line with the p values that are calculated.
		

\end{itemize}
	
\newpage
	
	
		% Table created by stargazer v.5.2.3 by Marek Hlavac, Social Policy Institute. E-mail: marek.hlavac at gmail.com
% Date and time: Sat, Mar 30, 2024 - 19:46:51
\begin{table}[!htbp] \centering 
	\caption{} 
	\label{} 
	\begin{tabular}{@{\extracolsep{5pt}} cccc} 
		\\[-1.8ex]\hline 
		\hline \\[-1.8ex] 
		& OR & 2.5 \% & 97.5 \% \\ 
		\hline \\[-1.8ex] 
		age & $0.993$ & $0.979$ & $1.006$ \\ 
		foreign\_born1 & $0.653$ & $0.381$ & $1.122$ \\ 
		as.factor(as.character(religiosity))secular & $0.540$ & $0.264$ & $1.086$ \\ 
		as.factor(as.character(religiosity))trad & $1.252$ & $0.560$ & $2.792$ \\ 
		as.factor(as.character(religiosity))u\_orthodox & $2.337$ & $1.159$ & $4.661$ \\ 
		sex1 & $0.459$ & $0.298$ & $0.704$ \\ 
		as.factor(education)high & $1.302$ & $0.493$ & $3.335$ \\ 
		as.factor(education)primary & $3.358$ & $0.861$ & $13.795$ \\ 
		as.factor(education)undergrad & $1.560$ & $0.586$ & $4.047$ \\ 
		as.factor(income)high & $0.545$ & $0.263$ & $1.131$ \\ 
		as.factor(income)low & $0.982$ & $0.537$ & $1.792$ \\ 
		as.factor(income)v\_high & $1.010$ & $0.348$ & $3.053$ \\ 
		as.factor(income)v\_low & $1.061$ & $0.598$ & $1.870$ \\ 
		left\_right & $1.501$ & $1.279$ & $1.767$ \\ 
		as.factor(ethnicity)mixed & $0.668$ & $0.303$ & $1.487$ \\ 
		as.factor(ethnicity)other & $1.284$ & $0.394$ & $4.492$ \\ 
		as.factor(ethnicity)sep & $1.237$ & $0.774$ & $1.983$ \\ 
		\hline \\[-1.8ex] 
	\end{tabular} 
\end{table} 
\newpage
\begin{table}[!htbp]
	\centering
	\caption{Regression Coefficients and Confidence Intervals}
	\label{tab:coefficients_CI}
	\begin{tabular}{lccc}
		\hline
		Variable & Coefficient & Lower CI & Upper CI \\
		\hline
		age & -0.007 & -0.021 & 0.006 \\
		foreign\_born1 & -0.426 & -0.965 & 0.115 \\
		religiosity (secular) & -0.617 & -1.332 & 0.083 \\
		religiosity (traditional) & 0.225 & -0.579 & 1.027 \\
		religiosity (ultra-orthodox) & 0.849 & 0.147 & 1.539 \\
		sex1 & -0.778 & -1.211 & -0.351 \\
		education (high) & 0.264 & -0.707 & 1.205 \\
		education (primary) & 1.211 & -0.150 & 2.624 \\
		education (undergrad) & 0.445 & -0.534 & 1.398 \\
		income (high) & -0.606 & -1.337 & 0.123 \\
		income (low) & -0.018 & -0.622 & 0.583 \\
		income (very high) & 0.010 & -1.057 & 1.116 \\
		income (very low) & 0.059 & -0.515 & 0.626 \\
		left\_right & 0.406 & 0.246 & 0.569 \\
		ethnicity (mixed) & -0.403 & -1.194 & 0.397 \\
		ethnicity (other) & 0.250 & -0.930 & 1.502 \\
		ethnicity (separate) & 0.213 & -0.257 & 0.685 \\
		\hline
	\end{tabular}
\end{table}

\begin{itemize}
	\item We can now run logit models for each category in order to see if the parallel line assumption holds
	\lstinputlisting[language=R, firstline=475, lastline=486]{KB_rep_cleaned.R}
	\item Results are shown in Table 6
	
\end{itemize}

% Table created by stargazer v.5.2.3 by Marek Hlavac, Social Policy Institute. E-mail: marek.hlavac at gmail.com
% Date and time: Sat, Mar 30, 2024 - 22:17:52
\begin{table}[!htbp] \centering 
	\caption{} 
	\label{} 
	\begin{adjustbox}{max width=\textwidth}
	\begin{tabular}{@{\extracolsep{5pt}}lccccccc} 
		\\[-1.8ex]\hline 
		\hline \\[-1.8ex] 
		& \multicolumn{7}{c}{\textit{Dependent variable:}} \\ 
		\cline{2-8} 
		\\[-1.8ex] & \multicolumn{7}{c}{arab\_acceptTEST == levels(arab\_acceptTEST)[i]} \\ 
		\\[-1.8ex] & (Relative) & (Friend) & (Neighbor) & (Coworker) & (Citizen) & (Visitor) & (None)\\ 
		\hline \\[-1.8ex] 
		age & $-$0.031 & $-$0.031$^{*}$ & 0.013 & $-$0.008 & 0.033$^{***}$ & 0.010 & $-$0.021$^{**}$ \\ 
		& (0.039) & (0.019) & (0.020) & (0.012) & (0.009) & (0.013) & (0.008) \\ 
		& & & & & & & \\ 
		foreign\_born1 & $-$18.790 & 1.953$^{***}$ & 0.184 & $-$0.500 & 0.088 & $-$0.018 & $-$0.356 \\ 
		& (7,232.651) & (0.736) & (0.713) & (0.512) & (0.362) & (0.479) & (0.346) \\ 
		& & & & & & & \\ 
		as.factor(as.character(religiosity))secular & 19.305 & 0.179 & $-$0.756 & $-$0.351 & 1.246$^{**}$ & 0.616 & $-$1.145$^{***}$ \\ 
		& (9,884.922) & (0.980) & (0.865) & (0.536) & (0.574) & (0.638) & (0.431) \\ 
		& & & & & & & \\ 
		as.factor(as.character(religiosity))trad & 18.834 & $-$1.224 & $-$0.871 & $-$0.881 & 0.851 & $-$0.072 & 0.185 \\ 
		& (9,884.922) & (1.226) & (1.036) & (0.712) & (0.619) & (0.739) & (0.450) \\ 
		& & & & & & & \\ 
		as.factor(as.character(religiosity))u\_orthodox & $-$0.118 & $-$0.842 & $-$2.310$^{**}$ & $-$1.820$^{***}$ & 0.734 & 0.085 & 0.649$^{*}$ \\ 
		& (11,153.780) & (1.102) & (1.051) & (0.578) & (0.556) & (0.625) & (0.381) \\ 
		& & & & & & & \\ 
		sex1 & 0.349 & 1.135$^{*}$ & $-$0.029 & 0.102 & 0.663$^{**}$ & 0.016 & $-$0.865$^{***}$ \\ 
		& (1.365) & (0.630) & (0.613) & (0.393) & (0.291) & (0.365) & (0.255) \\ 
		& & & & & & & \\ 
		as.factor(education)high & 18.275 & $-$1.529 & 1.073 & $-$0.881 & 0.810 & 15.536 & $-$0.305 \\ 
		& (13,139.300) & (0.940) & (1.332) & (0.721) & (0.743) & (861.706) & (0.581) \\ 
		& & & & & & & \\ 
		as.factor(education)primary & 1.839 & $-$17.209 & $-$15.837 & $-$1.165 & 0.800 & 14.892 & 0.618 \\ 
		& (19,616.320) & (1,415.656) & (3,812.976) & (1.257) & (0.945) & (861.706) & (0.779) \\ 
		& & & & & & & \\ 
		as.factor(education)undergrad & $-$0.556 & $-$3.270$^{**}$ & 1.078 & $-$0.556 & 0.775 & 15.683 & $-$0.207 \\ 
		& (14,362.630) & (1.318) & (1.322) & (0.725) & (0.747) & (861.706) & (0.587) \\ 
		& & & & & & & \\ 
		as.factor(income)high & 0.180 & 0.838 & 1.891 & $-$0.385 & $-$0.307 & $-$0.046 & $-$0.452 \\ 
		& (1.742) & (1.056) & (1.175) & (0.643) & (0.491) & (0.688) & (0.490) \\ 
		& & & & & & & \\ 
		as.factor(income)low & $-$0.375 & 1.481 & $-$15.752 & $-$0.161 & $-$0.148 & $-$0.731 & 0.193 \\ 
		& (1.586) & (0.962) & (1,652.911) & (0.531) & (0.375) & (0.597) & (0.364) \\ 
		& & & & & & & \\ 
		as.factor(income)v\_high & $-$17.820 & $-$0.230 & $-$16.210 & $-$1.130 & 0.497 & 0.499 & $-$0.276 \\ 
		& (16,584.300) & (1.706) & (4,497.914) & (1.152) & (0.682) & (0.899) & (0.695) \\ 
		& & & & & & & \\ 
		as.factor(income)v\_low & 0.230 & 0.740 & 1.808 & $-$0.076 & $-$0.928$^{**}$ & 0.345 & 0.180 \\ 
		& (1.548) & (0.977) & (1.137) & (0.506) & (0.383) & (0.483) & (0.339) \\ 
		& & & & & & & \\ 
		left\_right & $-$0.651 & $-$0.759$^{***}$ & $-$0.580$^{**}$ & $-$0.163 & $-$0.039 & 0.334$^{**}$ & 0.279$^{***}$ \\ 
		& (0.449) & (0.216) & (0.240) & (0.140) & (0.106) & (0.152) & (0.095) \\ 
		& & & & & & & \\ 
		as.factor(ethnicity)mixed & 1.456 & 1.301 & $-$0.163 & $-$1.118 & 0.701 & $-$0.196 & $-$0.495 \\ 
		& (1.788) & (0.946) & (0.983) & (0.815) & (0.533) & (0.820) & (0.501) \\ 
		& & & & & & & \\ 
		as.factor(ethnicity)other & $-$17.986 & 0.024 & 0.944 & $-$16.232 & $-$14.705 & 1.234 & 0.225 \\ 
		& (20,241.190) & (1.304) & (1.049) & (1,205.275) & (726.996) & (0.842) & (0.794) \\ 
		& & & & & & & \\ 
		as.factor(ethnicity)sep & 0.500 & 0.650 & $-$0.606 & $-$0.837$^{*}$ & 0.032 & 0.172 & 0.217 \\ 
		& (1.574) & (0.709) & (0.773) & (0.432) & (0.316) & (0.412) & (0.278) \\ 
		& & & & & & & \\ 
		Constant & $-$37.035 & 1.336 & $-$1.921 & 1.242 & $-$4.108$^{***}$ & $-$20.199 & $-$0.216 \\ 
		& (16,442.410) & (1.849) & (2.076) & (1.292) & (1.161) & (861.707) & (0.945) \\ 
		& & & & & & & \\ 
		\hline \\[-1.8ex] 
		Observations & 375 & 375 & 375 & 375 & 375 & 375 & 375 \\ 
		Log Likelihood & $-$13.329 & $-$51.361 & $-$43.917 & $-$107.725 & $-$174.212 & $-$116.887 & $-$212.732 \\ 
		Akaike Inf. Crit. & 62.658 & 138.721 & 123.835 & 251.451 & 384.425 & 269.775 & 461.465 \\ 
		\hline 
		\hline \\[-1.8ex] 
		\textit{Note:}  & \multicolumn{7}{r}{$^{*}$p$<$0.1; $^{**}$p$<$0.05; $^{***}$p$<$0.01} \\ 
	\end{tabular} 
	\end{adjustbox}
\end{table} 

\newpage

\begin{itemize}
	\item We can see that the parallel line assumption \textbf{does not appear to hold}, as we have sign changes as we move along the acceptance categories
	\item For example, in the age column, a one unit change in age is associated on average, with a 0.031 decrease in the the log odds of accepting a PCR as a relative, holding all other variables constant. 
	\item Moving along to the neighbor logit regression, we can see that  a one unit change in age is associated on average, with a 0.013 increase in the the log odds of accepting a PCR as a neighbor, holding all other variables constant.  
	\item Next, we see that for the coworker logit regression, a one unit change in age is associated on average, with a 0.008 decrease in the the log odds of accepting a PCR as a coworker, holding all other variables constant.
	\item These results are confusing, because if this were a truly ordinal scale we would see that these coefficients would move in the same direction. Based on just the 'relative' logit regression, would imply that older participants are less accepting of PCR. We would expect to see this effect consistent through all the logit regressions. Instead, we see sign flips throughout implying that the variable may not be truly ordinal.
	\item This effect is repeated throughout the other variables as well.
	\item In this case, we can run an unordered model to see if this is a better fit
\end{itemize}

	\subsubsection{Unordered Multinomial Model}
	\begin{itemize}
		\item First I converted the outcome variable to an unordered factor variable, set the reference category to 'none,' and ran the unordered multinomial regression. Table 7 shows the log odds regression results and Table 8 shows the coefficients converted to odds ratios.
		\lstinputlisting[language=R, firstline=551, lastline=580]{KB_rep_cleaned.R}		
	\end{itemize}


\begin{table}[!htbp] \centering 
	\caption{} 
	\label{} 
	\footnotesize
	\begin{adjustbox}{max width=\textwidth}
	\begin{tabular}{@{\extracolsep{5pt}}lcccccc} 
		\\[-1.8ex]\hline 
		\hline \\[-1.8ex] 
		& \multicolumn{6}{c}{\textit{Dependent variable:}} \\ 
		\cline{2-7} 
		\\[-1.8ex] & relative & friend & neighbor & coworker & citizen & visitor \\ 
		\\[-1.8ex] & (1) & (2) & (3) & (4) & (5) & (6)\\ 
		\hline \\[-1.8ex] 
		age & $-$0.024 & $-$0.019 & 0.017 & 0.003 & 0.036$^{***}$ & 0.021 \\ 
		& (0.041) & (0.020) & (0.021) & (0.014) & (0.011) & (0.014) \\ 
		& & & & & & \\ 
		foreign\_born1 & $-$18.421$^{***}$ & 2.039$^{**}$ & 0.558 & $-$0.178 & 0.367 & 0.188 \\ 
		& (0.00000) & (0.792) & (0.769) & (0.573) & (0.425) & (0.522) \\ 
		& & & & & & \\ 
		sex1 & 1.153 & 1.802$^{***}$ & 0.734 & 0.715$^{*}$ & 1.063$^{***}$ & 0.455 \\ 
		& (1.412) & (0.668) & (0.650) & (0.429) & (0.327) & (0.389) \\ 
		& & & & & & \\ 
		left\_right & $-$1.078$^{**}$ & $-$1.063$^{***}$ & $-$0.881$^{***}$ & $-$0.422$^{***}$ & $-$0.242$^{*}$ & 0.130 \\ 
		& (0.489) & (0.243) & (0.270) & (0.162) & (0.124) & (0.162) \\ 
		& & & & & & \\ 
		as.factor(as.character(religiosity))secular & 14.080$^{***}$ & 0.718 & $-$0.120 & 0.353 & 1.831$^{***}$ & 1.310$^{*}$ \\ 
		& (0.923) & (1.022) & (0.938) & (0.604) & (0.635) & (0.687) \\ 
		& & & & & & \\ 
		as.factor(as.character(religiosity))trad & 12.588$^{***}$ & $-$1.617 & $-$1.271 & $-$1.056 & 0.609 & $-$0.126 \\ 
		& (0.957) & (1.255) & (1.110) & (0.763) & (0.660) & (0.767) \\ 
		& & & & & & \\ 
		as.factor(as.character(religiosity))u\_orthodox & $-$7.714$^{***}$ & $-$1.521 & $-$2.830$^{***}$ & $-$2.047$^{***}$ & 0.330 & $-$0.154 \\ 
		& (0.000) & (1.140) & (1.098) & (0.619) & (0.591) & (0.652) \\ 
		& & & & & & \\ 
		as.factor(education)high & 10.577$^{***}$ & $-$1.171 & 1.032 & $-$0.711 & 0.620 & 27.344$^{***}$ \\ 
		& (1.182) & (1.053) & (1.465) & (0.801) & (0.825) & (0.476) \\ 
		& & & & & & \\ 
		as.factor(education)primary & $-$1.296$^{***}$ & $-$46.280 & $-$23.232$^{***}$ & $-$1.662 & 0.129 & 26.348$^{***}$ \\ 
		& (0.00000) & (NA) & (0.000) & (1.340) & (1.017) & (0.894) \\ 
		& & & & & & \\ 
		as.factor(education)undergrad & $-$8.259$^{***}$ & $-$3.053$^{**}$ & 0.742 & $-$0.581 & 0.518 & 27.443$^{***}$ \\ 
		& (0.00000) & (1.407) & (1.460) & (0.812) & (0.827) & (0.482) \\ 
		& & & & & & \\ 
		as.factor(income)high & 1.071 & 1.510 & 2.335$^{*}$ & 0.296 & 0.207 & 0.291 \\ 
		& (1.797) & (1.147) & (1.248) & (0.733) & (0.589) & (0.756) \\ 
		& & & & & & \\ 
		as.factor(income)low & $-$0.103 & 1.492 & $-$31.733$^{***}$ & $-$0.120 & $-$0.163 & $-$0.766 \\ 
		& (1.619) & (1.015) & (0.000) & (0.584) & (0.426) & (0.629) \\ 
		& & & & & & \\ 
		as.factor(income)v\_high & $-$11.807$^{***}$ & $-$0.230 & $-$17.987$^{***}$ & $-$0.893 & 0.594 & 0.655 \\ 
		& (0.00000) & (1.822) & (0.00000) & (1.263) & (0.800) & (0.987) \\ 
		& & & & & & \\ 
		as.factor(income)v\_low & 0.376 & 0.793 & 1.757 & $-$0.062 & $-$0.799$^{*}$ & 0.190 \\ 
		& (1.592) & (1.024) & (1.176) & (0.551) & (0.426) & (0.514) \\ 
		& & & & & & \\ 
		as.factor(ethnicity)mixed & 1.622 & 1.688 & 0.118 & $-$0.596 & 0.965 & 0.167 \\ 
		& (1.839) & (1.036) & (1.071) & (0.890) & (0.624) & (0.871) \\ 
		& & & & & & \\ 
		as.factor(ethnicity)other & $-$16.002$^{***}$ & $-$0.723 & 0.054 & $-$30.247$^{***}$ & $-$27.855$^{***}$ & 0.822 \\ 
		& (0.00000) & (1.498) & (1.232) & (0.000) & (0.000) & (0.937) \\ 
		& & & & & & \\ 
		as.factor(ethnicity)sep & 0.132 & 0.337 & $-$0.809 & $-$0.887$^{*}$ & $-$0.112 & 0.016 \\ 
		& (1.529) & (0.737) & (0.804) & (0.470) & (0.353) & (0.435) \\ 
		& & & & & & \\ 
		Constant & $-$21.131$^{***}$ & 2.811 & 0.161 & 2.361 & $-$2.519$^{*}$ & $-$30.769$^{***}$ \\ 
		& (1.182) & (2.054) & (2.326) & (1.477) & (1.320) & (1.046) \\ 
		& & & & & & \\ 
		\hline \\[-1.8ex] 
		Akaike Inf. Crit. & 1,090.955 & 1,090.955 & 1,090.955 & 1,090.955 & 1,090.955 & 1,090.955 \\ 
		\hline 
		\hline \\[-1.8ex] 
		\textit{Note:}  & \multicolumn{6}{r}{$^{*}$p$<$0.1; $^{**}$p$<$0.05; $^{***}$p$<$0.01} \\ 
	\end{tabular} 
	\end{adjustbox} 
\end{table}

\newpage

\begin{table}[!htbp]
	\centering 
	\caption{} 
	\label{} 
	\begin{adjustbox}{max width=\textwidth}
		\begin{tabular}{@{\extracolsep{5pt}}lcccccc} 
			\hline 
			\hline \\[-1.8ex] 
			& \multicolumn{6}{c}{\textit{Dependent variable:}} \\ 
			\cline{2-7} 
			\\[-1.8ex] & relative & friend & neighbor & coworker & citizen & visitor \\ 
			\\[-1.8ex] & (1) & (2) & (3) & (4) & (5) & (6)\\ 
			\hline \\[-1.8ex] 
			(Intercept) & 6.651812e-10 & 16.62011 & 1.174264 & 10.6033 & 0.08055344 & 4.337311e-14 \\
			age & 0.9766339 & 0.9813665 & 1.0170968 & 1.0027272 & 1.0362798 & 1.0211608 \\
			foreign\_born1 & 9.993412e-09 & 7.680720 & 1.746663 & 0.8367114 & 1.442817 & 1.206234 \\
			sex1 & 3.167965 & 6.061605 & 2.082368 & 2.044595 & 2.895577 & 1.576944 \\
			left\_right & 0.3402059 & 0.3454747 & 0.4144849 & 0.6555768 & 0.7852415 & 1.1388934 \\
			as.factor(as.character(religiosity))secular & 1.302773e+06 & 2.051171 & 0.8867322 & 1.42354 & 6.238936 & 3.706946 \\
			as.factor(as.character(religiosity))trad & 2.929564e+05 & 0.1985388 & 0.2804708 & 0.3477121 & 1.837836 & 0.8815897 \\
			as.factor(as.character(religiosity))u\_orthodox & 0.000446421 & 0.2185649 & 0.05903657 & 0.1291617 & 1.390973 & 0.8576777 \\
			as.factor(education)high & 39234.80 & 0.3101697 & 2.807297 & 0.4909893 & 1.858677 & 750693100000 \\
			as.factor(education)primary & 0.2736024 & 7.959215e-21 & 8.13683e-11 & 0.1896932 & 1.137379 & 2.771624e+11 \\
			as.factor(education)undergrad & 0.0002588371 & 0.04723659 & 2.100926 & 0.5594632 & 1.678288 & 8.285699e+11 \\
			as.factor(income)high & 2.919292 & 4.528083 & 10.32993 & 1.34446 & 1.229912 & 1.337493 \\
			as.factor(income)low & 0.9019111 & 4.446884 & 1.654761e-14 & 0.886939 & 0.8492521 & 0.4649023 \\
			as.factor(income)v\_high & 7.451886e-06 & 0.7944283 & 1.542646e-08 & 0.4094867 & 1.81096 & 1.92551 \\
			as.factor(income)v\_low & 1.4566613 & 2.2092725 & 5.7967432 & 0.9400315 & 0.4495974 & 1.2089442 \\
			as.factor(ethnicity)mixed & 5.06502 & 5.408902 & 1.125156 & 0.550868 & 2.624276 & 1.181947 \\
			as.factor(ethnicity)other & 1.123243e-07 & 0.4855062 & 1.055101 & 7.309264e-14 & 7.992025e-13 & 2.275707 \\
			as.factor(ethnicity)sep & 1.1416274 & 1.4000588 & 0.4451049 & 0.4117924 & 0.8942012 & 1.0161241 \\
			\hline 
			\hline \\[-1.8ex] 
			\textit{Note:}  & \multicolumn{6}{r}{$^{*}$p$<$0.1; $^{**}$p$<$0.05; $^{***}$p$<$0.01} \\ 
		\end{tabular} 
	\end{adjustbox} 
\end{table} 


\begin{itemize}
	\item Next we can look at the prediction accuracy
	\lstinputlisting[language=R, firstline=605, lastline=621]{KB_rep_cleaned.R}	
	\item The results are shown in Table 9
\end{itemize}

\newpage
% latex table generated in R 4.3.1 by xtable 1.8-4 package
% Sun Mar 31 02:06:54 2024
\begin{table}[!htbp]
	\centering
	\caption{Prediction Accuracy}
	\label{tab:confusion_matrix}
	\begin{tabular}{rrrrrrrrr}
		\hline
		& \multicolumn{7}{c}{\textbf{Predictions}} \\
		\cline{2-8}
		\multirow{2}{*}{\textbf{Actual Outcomes}} & none & relative & friend & neighbor & coworker & citizen & visitor & \textbf{Sum} \\
		& & & & & & & & \\
		\hline
		citizen   & 40.00 & 0.00 & 2.00 & 1.00 & 2.00 & 38.00 & 0.00 & 83.00 \\
		coworker  & 19.00 & 0.00 & 2.00 & 1.00 & 7.00 & 8.00 & 0.00 & 37.00 \\
		friend    & 5.00  & 0.00 & 8.00 & 1.00 & 1.00 & 3.00 & 1.00 & 19.00 \\
		neighbor  & 5.00  & 0.00 & 0.00 & 4.00 & 1.00 & 5.00 & 0.00 & 15.00 \\
		none      & 157.00& 0.00 & 2.00 & 1.00 & 3.00 & 13.00& 1.00 & 177.00 \\
		relative  & 1.00  & 1.00 & 1.00 & 0.00 & 0.00 & 1.00 & 0.00 & 4.00 \\
		visitor   & 26.00 & 0.00 & 0.00 & 1.00 & 2.00 & 10.00& 1.00 & 40.00 \\
		\hline
		\textbf{Sum} & 253.00 & 1.00 & 15.00 & 9.00 & 16.00 & 78.00 & 3.00 & 375.00 \\
		\hline
	\end{tabular}
\end{table}

\begin{itemize}
	\item We can see in this table that the model tends to overpredict the "none" outcome and underpredicts the visitor outcome.
	\item Furthermore, we can see that the results of the unordered multinomial regression are very messy and hard to interpret due to the number of outcomes possible and the amount of predictors used in the model. 
	\item To make this slightly easier to look at, I also decided to run this as a binary variable, which the authors do in their second model of cooperation. This counts everyone who said they would accept a PCI as a coworker or closer as low exclusionary preference and all other options as high exclusionary preference. Results are shown in Table 9 (log odds) and Table 10 (odds ratios)
	\lstinputlisting[language=R, firstline=656, lastline=661]{KB_rep_cleaned.R}	
\end{itemize}


% Table created by stargazer v.5.2.3 by Marek Hlavac, Social Policy Institute. E-mail: marek.hlavac at gmail.com
% Date and time: Sun, Mar 31, 2024 - 13:05:35
\begin{table}[!htbp] \centering 
	\caption{} 
	\label{} 
	\scriptsize
	\begin{adjustbox}{max width=\textwidth}
	
	\begin{tabular}{@{\extracolsep{2pt}}lc} 
		\\[-1.8ex]\hline 
		\hline \\[-1.8ex] 
		& \multicolumn{1}{c}{\textit{Dependent variable:}} \\ 
		\cline{2-2} 
		\\[-1.8ex] & arab\_reject\_binary \\ 
		\hline \\[-1.8ex] 
		age & 0.014 \\ 
		& (0.010) \\ 
		& \\ 
		foreign\_born1 & $-$0.227 \\ 
		& (0.392) \\ 
		& \\ 
		sex1 & $-$0.565$^{*}$ \\ 
		& (0.320) \\ 
		& \\ 
		left\_right & 0.588$^{***}$ \\ 
		& (0.124) \\ 
		& \\ 
		as.factor(as.character(religiosity))secular & 0.500 \\ 
		& (0.468) \\ 
		& \\ 
		as.factor(as.character(religiosity))trad & 1.169$^{**}$ \\ 
		& (0.577) \\ 
		& \\ 
		as.factor(as.character(religiosity))u\_orthodox & 2.173$^{***}$ \\ 
		& (0.512) \\ 
		& \\ 
		as.factor(education)high & 1.082$^{*}$ \\ 
		& (0.617) \\ 
		& \\ 
		as.factor(education)primary & 2.606$^{**}$ \\ 
		& (1.261) \\ 
		& \\ 
		as.factor(education)undergrad & 1.383$^{**}$ \\ 
		& (0.639) \\ 
		& \\ 
		as.factor(income)high & $-$0.778 \\ 
		& (0.526) \\ 
		& \\ 
		as.factor(income)low & $-$0.308 \\ 
		& (0.466) \\ 
		& \\ 
		as.factor(income)v\_high & 1.227 \\ 
		& (0.988) \\ 
		& \\ 
		as.factor(income)v\_low & $-$0.535 \\ 
		& (0.437) \\ 
		& \\ 
		as.factor(ethnicity)mixed & 0.191 \\ 
		& (0.545) \\ 
		& \\ 
		as.factor(ethnicity)other & 0.676 \\ 
		& (0.857) \\ 
		& \\ 
		as.factor(ethnicity)sep & 0.544 \\ 
		& (0.362) \\ 
		& \\ 
		Constant & $-$3.922$^{***}$ \\ 
		& (1.152) \\ 
		& \\ 
		\hline \\[-1.8ex] 
		Akaike Inf. Crit. & 329.296 \\ 
		\hline 
		\hline \\[-1.8ex] 
		\textit{Note:}  & \multicolumn{1}{r}{$^{*}$p$<$0.1; $^{**}$p$<$0.05; $^{***}$p$<$0.01} \\ 
	\end{tabular} 
\end{adjustbox}
\end{table} 

% latex table generated in R 4.3.1 by xtable 1.8-4 package
% Sun Mar 31 13:09:21 2024
\begin{table}[ht]
	\centering
	\caption{} 
	\label{} 
	\begin{tabular}{rr}
		\hline
		& ORUnordBin \\ 
		\hline
		(Intercept) & 0.02 \\ 
		age & 1.01 \\ 
		foreign\_born1 & 0.80 \\ 
		sex1 & 0.57 \\ 
		left\_right & 1.80 \\ 
		as.factor(as.character(religiosity))secular & 1.65 \\ 
		as.factor(as.character(religiosity))trad & 3.22 \\ 
		as.factor(as.character(religiosity))u\_orthodox & 8.78 \\ 
		as.factor(education)high & 2.95 \\ 
		as.factor(education)primary & 13.54 \\ 
		as.factor(education)undergrad & 3.99 \\ 
		as.factor(income)high & 0.46 \\ 
		as.factor(income)low & 0.73 \\ 
		as.factor(income)v\_high & 3.41 \\ 
		as.factor(income)v\_low & 0.59 \\ 
		as.factor(ethnicity)mixed & 1.21 \\ 
		as.factor(ethnicity)other & 1.97 \\ 
		as.factor(ethnicity)sep & 1.72 \\ 
		\hline
	\end{tabular}
\end{table}

\newpage
\begin{itemize}
	\item These results make significantly more sense and are easier to interpret. 
	\item We can also check the predictions, shown in Table 12
	\lstinputlisting[language=R, firstline=684, lastline=699]{KB_rep_cleaned.R}	
	
\end{itemize}

\newpage

% latex table generated in R 4.3.1 by xtable 1.8-4 package
% Sun Mar 31 13:20:37 2024
\begin{table}[ht]
	\centering
	\caption{} 
	\label{} 
	\begin{tabular}{rrrr}
		\hline
		& 0 & 1 & Sum \\ 
		\hline
		0 & 27.00 & 48.00 & 75.00 \\ 
		1 & 14.00 & 286.00 & 300.00 \\ 
		Sum & 41.00 & 334.00 & 375.00 \\ 
		\hline
	\end{tabular}
\end{table}


\begin{itemize}
	\item This prediction table implies that the model correctly predicts exclusionary attitudes relatively well (286/300) but not non-exclusionary attitudes (27/75).
\end{itemize}



\subsection{Overall Findings of My Twist}

\begin{itemize}
	\item The authors found that "In this article, we have shown that social distance, a measure of exclusionary preferences, is strongly predictive of cooperation in a public goods game" (p. 753).
	\item The authors also state "Political ideology, education, and religiosity all appear to be strongly related to social distance, with more right-wing, more religious, and less-educated subjects expressing more exclusionary preferences" (p. 749).
	\item My results from Table 3 indicate that at the $p < 0.05$ level, right-wing and ultra-Orthodox participants were associated with a positive increase in the odds of holding higher exclusionary attitudes on an ordinal scale. At the $p < 0.01$ level, primary education, compared to graduate education, was associated with an increase in exclusionary attitudes. Therefore, less educated participants were associated with a positive increase in the odds of holding higher exclusionary attitudes on an ordinal scale. Furthermore, female participants were associated, on average, with a decrease in the log odds of holding exclusionary attitudes.
	\item All of this together, I would say that the conclusion that the 'results are substantively similar' is partially justified. Based on the conclusions drawn by the authors about which factors are significant in their overall analysis, the results are justified. However, this is based on the assumption that the social desirability scale is truly ordinal and that this simplification is not impactful of the overall results.
	\item Based on the results of the parallel line assumption test (Table 6), we can see that this assumption does not appear to hold.
	\item Within the results of the unordered multinomial model, we can see that many of the predictors appear to have abnormally strong associations with social closeness preferences. 
	\item However, the predictive power of this model seems quite good for predicting the 'none' outcome for not accepting PCI at all. The other categories do not seem as strong though. 
	\item Overall, it seems as though all of the models I tested seem to be relatively good at predicting a lack of acceptance towards PCI. However, each model has its own setback: the OLS model such that the outcome variable is not continuous in nature and the cut points of the ordinal model imply that this model might be skewed; the ordinal model does not pass the parallel line assumption test; and the unordered model produces strangely high odds ratios and has quite poor predictive power. 
	\item The conclusion that I have drawn from this is that the scale from which the major conclusions were drawn from might be skewed, leading to abnormal results. This is supported by Weinfurt and Moghaddam (2001), who state the difficulties of using the social closeness scale as an ordinal measure in various cultural contexts
\end{itemize}

\subsection{References}
\begin{list}{}{\setlength{\leftmargin}{2.5em}\setlength{\itemindent}{-2.5em}}
	\item[] Enos, R. D., \& Gidron, N. (2018). Exclusion and Cooperation in Diverse Societies: Experimental Evidence from Israel. \textit{American Political Science Review}, \textit{112}(4), 742–757. https://doi.org/10.1017/S0003055418000266
	\item[] Weinfurt, K. P., \& Moghaddam, F. M. (2001). Culture and Social Distance: A Case Study of Methodological Cautions. \textit{The Journal of Social Psychology}, \textit{141}(1), 101–110. https://doi.org/10.1080/00224540109600526
\end{list}



\end{document}